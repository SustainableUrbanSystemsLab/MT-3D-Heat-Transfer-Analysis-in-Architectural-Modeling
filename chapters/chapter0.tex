%!TEX root = ../thesis.tex
\chapter{Introduction and Background}
This chapter provides a detailed overview of the thesis, focusing on the problem and the motivation for undertaking this research. It highlights the focus of this project and outlines the research aim, objective, and relevance in practice.

Climate change is one of the most significant challenges of our time, resulting in severe consequences for the environment, society, and the economy. The alarming increase in global land temperature, which has risen by approximately 40\% \cite{glb} in recent years, highlights the importance of addressing this issue. At the same time, buildings contribute substantially to carbon emissions, accounting for approximately 40 gigatons of CO2 from only operational carbon per year. To prevent further escalation of climate change and maintain the global warming limit of 1.5°C. To achieve the goal of maintaining global warming, the emissions must be reduced by 10 gigatons of CO2 annually \cite{ipcc}.

Focusing on the architectural and construction fields, architects and engineers refer to \gls{ASHRAE} standards to comply with the minimum required insulation based on the location of a project. 
In adhering to these standards, there is the potential to optimize the selection of materials toward lower energy demand and to increase thermal comfort. However, modeling 3D heat transfer for applications in buildings is a complex and long process for modelers due to the lack of available free software. 

This thesis aims to bridge the gap between providing architects with easy-to-use 3D heat transfer software that is integrated into architect design software, such as \textit{Rhino\, \&\, \gls{GH}}. Here, we used \textit{\gls{OF} / \gls{GH}} to construct an envelope segment from \textit{ISO 10211:2007}
\cite{ISO}, then calculate heat transfer, and assess whether the validated case complies with our results. 



\section{Problem Statement}
Simulating 3D heat transfer is challenging due to the complexity of the differential equations as a result of the heat movement in 3D spaces. This complexity creates obstacles in achieving accurate simulations, which are critical in several applications. In the architectural field, the importance of 3D heat transfer simulations is significant to analyze thermal comfort and thermal behavior on materials. The 3D simulation also has a direct impact on building design, affecting everything from the building envelope to material choices and even the comfort and safety of occupants. The simulation also offers a detailed description of the modeled state, where thermal bridges and complex geometries are represented precisely.
However, existing software and simulation tools in architecture are generally limited to 2D heat transfer simulations, which cannot fully show the complexities represented in 3D thermal dynamics. Tools such as HEAT2, HTFlux, and THERM, widely used in the industry, provide only a partial view of the thermal environment, leading to potential inaccuracies in architectural planning or retrofitting.

Due to these limitations, the research question is the following: What approaches can be used to design and validate a workflow that simulates 3D heat transfer in architectural contexts? How can this workflow be seamlessly integrated into existing architectural modeling software? These questions aim to bridge the gap in current simulation technology by investigating the creation of a more advanced and reliable 3D simulation solution for architectural projects.

\section{Research Hypotheses}
This research is centered on the following hypotheses.
The development and validation of an integrated 3D heat transfer workflow to reduce errors in architectural thermal analysis, offering an easier approach to architectural decision-making.

\subsection{Applied Relevance}
\begin{itemize}
    \item The capacity to simulate both steady-state and transient heat transfer in \gls{3D} using \gls{OF} is crucial for accurate modeling in architectural contexts. The simulations are validated with the \gls{ISO} and experimental design data, to ensure reliability in the results.
    \item Integrating OpenFOAM with Rhino and \gls{GH} facilitates the inclusion of \gls{CFD} into architectural modeling. This connection simplifies workflows, making it easier for architects to incorporate complex heat transfer analyses into their design processes.
\end{itemize}

\subsection{Functional Benefits}
\begin{itemize}
    \item The architectural field lacks the workflows directly integrated into the architectural modeling software and is cost-efficient to simulate \gls{3D} heat transfer.
    \item The automation of writing the OF text files reduces tremendous effort and time
    \item This workflow is open-source, allowing architects to model, run, and iterate simulations without any cost, providing feedback at every stage of design or retrofitting.

    \item Eliminates the need to download additional software, rebuild models, reassign materials, or wait for feedback from a separate department.
\end{itemize}

\subsection{Practical Applications}
\begin{itemize}
\item This project presents a seamless workflow that contributes to design optimizations, reducing carbon emissions from both operations and construction materials.
\item No software costs, reduced project expenses. Eliminates the need for multiple tools, reducing complexity and time delays.
\item The open-source workflow is accessible and motivates and encourages any designer to reduce operational and embodied carbon.


\end{itemize}
\section{Research Aim \& Objectives}
\subsection{Research Aim}
The main objective of the research is to develop and validate a comprehensive 3D heat transfer simulation workflow integrated into architectural modeling processes. To achieve this research goal, the following specific objectives have been established.

\subsection{Research Objectives}
\begin{itemize}
    \item Select a suitable case study to validate the simulation results.
    \item Develop a connection between Rhino and automate the (\gls{OF}) case in Grasshopper (\gls{GH}) to streamline the architectural modeling workflow. 
    \item  Create a 3D heat transfer simulation workflow using OpenFOAM.
    \item Ensure the accuracy of the 3D simulation results through comprehensive validation.
    \item Test the developed workflow by incorporating real-time heat flux sensor readings.
 
\end{itemize}







\section{Thesis Outline}
This thesis is structured as follows:
\begin{itemize}
    \item \textit{Chapter 2} Showcases the history of heat transfer in buildings, literature review, heat transfer processes, and computational fluid dynamics.
    \item \textit{Chapter 3} illustrates three approaches to finding the heat flux of a brick wall; manual calculation, sensor reading, and 2D heat transfer simulation. These approaches are presented to highlight the shortcomings of current resources. 
    \item \textit{Chapter 4} Focuses on the methodology of the 3D heat transfer workflow, identifies an appropriate validation case study, and discusses the solver and process in detail.
    \item \textit{Chapter 5} Ensure the compliance of the objectives and provide an outlook with suggestions for an expansion of this project.
\end{itemize}