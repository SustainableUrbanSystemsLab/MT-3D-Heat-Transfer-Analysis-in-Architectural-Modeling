\chapter{Heat Transfer in Buildings}



The capabilities of current software packages are specific to remodeling one wall or section in separate software, rather than integrating it into the design process using design software packages. Examples of current 2D heat transfer software are THERM and HTflux, which typically focus on remodeling specific wall sections rather than integrating it into the overall design process. Table \ref{tab:heat_transfer_software} is a detailed description of current 2D and 3D heat transfer software with their types and limitations. 

Traditional building design processes often rely on simplified 2D heat transfer models that fail to capture the precise thermal interactions that occur in the real world. This limitation affects the accuracy of performance predictions and affects the identification of optimal design solutions for energy efficiency. 
3D simulations offer a more realistic representation of thermal transfer, especially if they account for both conduction and convection. 
In addition, they have significant potential for cost reduction and carbon emission mitigation. 


%The impact of this research is significant on energy efficiency, reduced operational carbon, and possibly reduced embodied carbon by adding the climate zone, and the tool will provide recommendations.

%\clearpage



\section{Literature Review}





%\section{Literature Review}
%We conducted a qualitative literature review to understand the current potential of 3D heat transfer software for architects. 
% First, a review was made to understand the capabilities of $HTflux$ and $THERM$ \cite{THERM}, which are the main heat transfer software for modelers. 

% HTFlux
Current tools are limited to 2D heat transfer, such as HTflux \cite{HTflux}, which is specialized software for simulations of two-dimensional heat and water vapor transport.
It uses the 2D Glaser method \cite{glaser1959graphisches} to calculate dew points, condensation, and evaporation in 2D structures. 
The software offers an easy-to-use interface, can import \gls{CAD} geometries, and uses a direct mapping method for accurate simulations. 
It provides various thermal analysis metrics (heat flow, U-value, $\psi$-value, etc.) and measures extreme temperature values. Two contributions stand out that discuss different 3D heat transfer elements and approaches by  \citeauthor{Yang} \cite{Yang}, and \citeauthor{COMSOL} \cite{COMSOL}, while one contribution presents 2D heat transfer using \gls{OF} \cite{kastner2020solving}.

\citeauthor{kastner2020solving} \cite{kastner2020solving} presented a case study that presents a thermal bridge analysis and 2D heat transfer with \gls{OF}, which is similar to our approach due to the use of \gls{OF}. The simulation is constructed from pre-processing, simulation, and post-processing. 
Based on the result of the used case, it is possible to use Rhino\ \&\ Grasshopper to calculate architectural heat transfer in the pre-and post-processing phases, but this requires additional research to achieve this goal. 
The authors experimented with a validation case retrieved from HTflux that was originally published in the ISO guide. 

The second approach is found from \citeauthor{Yang} \cite{Yang} which experimented with 3D heat transfer following a different method, which is to develop a mesh using lumped hexahedral elements and employing ray/triangle intersection techniques for an accurate geometric representation of the building. 
It applies an energy balance to each element and integrates a system of ordinary differential equations to obtain spatio-temporal indoor temperature and relative humidity fields. 
However, there are several limitations, such as the inability to explicitly solve flow fields, which may limit its accuracy in capturing complex airflow patterns. 
Furthermore, idealized thermal conditions do not include real-world conditions and do not accurately represent the impact of the thermal mass of the floor. 


The third approach of \citeauthor{COMSOL} \cite{COMSOL} used COMSOL / Multiphysics to simulate heat transfer in buildings and has been validated. There were minor differences in the results because of the convective heat transfer coefficient. 
The main impediments with COMSOL / Multiphysics are the cost and the need to use software that is not integrated into the design software.
However, the authors' work is valuable in terms of offering heat transfer simulation focusing on buildings.  %cost

The final research by \citeauthor{litrev2} \cite{litrev2} is directly related to the aim of this thesis. 
It uses computational fluid dynamics (CFD) simulation of convective heat transfer, but specifically in Japanese vernacular architecture, focusing on “machiya” buildings. These traditional structures offer passive strategies for maintaining indoor comfort with minimal assistance from HVAC. The purpose of the study is to validate and develop a methodology for simulating convective heat transfer using high-resolution 3D steady \gls{RANS} CFD simulations. Through comparison and evaluation of different \gls{RANS} models and boundary layer modeling approaches, the researchers identify suitable methods to predict convective heat transfer coefficients (\gls{CHTC}) and flow fields. Validation against wind tunnel experiments on heated cubes in turbulent channel flow confirms the effectiveness of selected models and approaches. The study contributes to the advancement of simulation methodologies for sustainable building design but did not meet our objective of automating the heat transfer workflow to simplify it for the user. 



\begin{landscape}
\begin{table}[htb]
    \centering
    \footnotesize
    \caption{Heat Transfer Simulation Software Overview.}
    \label{tab:heat_transfer_software}
    \begin{tabular}{ll>{\raggedright}p{5.3cm}lp{1cm}p{1cm}p{1.5cm}p{1cm}>{\raggedright}p{4cm}r} 
        \toprule
        Software & Type & Overview & Price & Rhino / Revit & Ease of Use & Complex Geometry & Export PV & Limitations & Source \\
        \midrule
        THERM     & 2D & Finite-element approach for complex geometries & \$0 & \ding{55} & \ding{51} & \ding{55} & \ding{55} & Limited to 2D, requires additional tools. & \cite{THERM} \\
        HTFLUX    & 2D & Heat and water vapor transport simulation & \$421/yr & \ding{55} & \ding{51} & \ding{55} & \ding{55} & Requires additional tool downloads. & \cite{HTflux} \\
        ENERGY2D  & 2D & Multiphysics program for conduction, convection, and radiation & \$0 & \ding{55} & \ding{55} & \ding{55} & \ding{55} & No user interface. Requires additional tool downloads. & \cite{energy2d} \\
        HEAT2     & 2D & Transient and steady-state heat transfer & \$800/yr & \ding{55} & \ding{51} & \ding{55} & \ding{55} & Requires additional tool downloads. & \cite{heat2} \\
        Kelvin    & 2D & Temperature, heat flux, and gradient simulation & NA & \ding{55} & \ding{51} & \ding{55} & \ding{55} & Requires additional tool downloads. & \cite{kelvin} \\
        Quickfield & 3D & Static and transient heat transfer analysis & Node-based & \ding{55} & \ding{51} & \ding{51} & \ding{55} & Limited post-processing, additional tools. & \cite{quickfield} \\
        Simscale  & 3D & Models heat transfer in solids and fluids through convection & Core-based & \ding{55} & \ding{55} & \ding{51} & \ding{55} & Cost-prohibitive, limited post-processing. & \cite{simscale} \\
        COMSOL    & 3D & Heat transfer by conduction, convection, and radiation & \$4000 & \ding{55} & \ding{55} & \ding{51} & \ding{55} & Cost-prohibitive, requires additional tools. & \cite{COMSOL} \\
        theseus   & 3D & FEM for steady-state and transient analysis & NA & \ding{55} & \ding{51} & \ding{51} & \ding{51} & Limited post-processing, cost-prohibitive. & \cite{theusus} \\
        Physibel (S)  & 3D & Examines entire buildings with 2D/3D components & \$2,359/yr & \ding{55} & \ding{55} & \ding{51} & \ding{55} & Cost-prohibitive, limited post-processing. & \cite{physibel} \\
        TAITHERM  & 3D & Simulates transient or steady-state heat transfer & NA & \ding{55} & \ding{51} & \ding{51} & \ding{55} & Cost-prohibitive, limited post-processing. & \cite{taitherm} \\
        \bottomrule
    \end{tabular}
\end{table}
\enlargethispage{1cm}

\end{landscape}

    


















% Heim
%\citet{Heim2005} presented a two-solution method to simulate a dynamic heat transfer problem with phase change materials (PCM). The objectives of the paper are to provide a comparison between two solutions for specific and latent heat transfer through the building components, which are the effective heat capacity method and the additional heat source method. There are several differences between the two methods, such as, in the effective heat capacity method, the heat capacity is considered a function of the melting and solidification phase change. However, in the additional heat source method, the phase change depends on the temperature of the internal heat gain flux. The study is carried out around the embedded PCM in the building structure to add additional latent heat and increase the thermal capacity of the material. Both methods achieved the objective with a slight difference in the results. However, the author indicates that the numerical models used in the ESP-r require further validation and refinement.  


%\clearpage




\section{Heat Transfer Processes}


This thesis presents a simulation based on conductive and convective heat transfer. 
Conductive heat transfer can be calculated using the heat diffusion equation \cite{bergman2011fundamentals}:
	
	\begin{equation} 
	\frac{\partial^2 T}{\partial x^2}+
	\frac{\partial^2 T}{\partial y^2}+
	\frac{\partial^2 T}{\partial z^2}+ 
	\frac{\dot{q}}{k}= \frac{1}{\alpha}\frac{\partial T}{\partial t} \text{, with } \alpha = \frac{k}{\rho c_p}\quad \\
	\end{equation}
	
	
Here, $x,y,$ and $z$ are the Cartesian coordinates, $\alpha$ is the thermal diffusivity, $\dot{q}$ is the energy generation rate per unit volume, and \glsfirst{cp} is the \glsentrylong{cp}.
 
To calculate convective heat transfer, the formula used accounts for both advection (the movement of heat with the fluid flow) and diffusion (the spatial variation of temperature): 

\begin{equation}
    \frac{\partial T}{\partial t} + \vb v \cdot \nabla T = \alpha \nabla^2 T + \frac{\dot{q}}{k},
\end{equation}

where $\mathbf{v}$ is the velocity vector of the fluid, $\alpha$ is the thermal diffusivity, $\nabla$ stands for the gradient operator, $\nabla^2$ represents the Laplacian operator, $\dot{q}$ is the energy generation rate per unit volume, and $k$ is the thermal conductivity \cite{bergman2011fundamentals}. This project utilizes the conjugate convective heat transfer approach, which is the heat transfer of the combination of solids and fluids. Here, the solids are solved by the steady-state Laplace equation, and the fluid flow is solved by the Navier-Stokes equations. Finally, the interfaces are matched to provide temperature distributions and heat flux \cite{Zhao2007}. Thus, performing 3D heat transfer in most architectural applications is challenging and cannot be solved by hand due to the complexity of the mathematics involved.



\section{Computational Fluid Dynamics}
We use \gls{OF} which stands for Open-source Field Operation and Manipulation and is a widely used open-source \gls{CFD} software package that allows engineers and researchers to simulate fluid flow, heat transfer, and other related phenomena using numerical methods. The project uses \gls{CFD} to calculate the heat flow gradient in the space. The selected \gls{OF} solver in this project is \gls{CHT} \textit{chtMultiRegionFoam} which is a solver capable of solving steady or transient fluid flow with solid heat conduction and conjugate heat transfer between regions, buoyancy effects, turbulence, reactions, and radiation modeling\footnote{Not used in this study.} \cite{cht}. A detailed description of the solver can be found in Chapter 3.


