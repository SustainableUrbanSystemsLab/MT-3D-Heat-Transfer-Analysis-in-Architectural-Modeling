%\clearpage
\chapter{Conclusion}
This thesis presents an automated integration between  \gls{GH}, Rhino, and \gls{OF} workflow to simulate 3D heat transfer in architectural applications. 
By utilizing \gls{OF}, this research fills a critical gap in current design practices and is capable of improving energy-efficient building design and data-driven material selection. 

This 3D heat transfer workflow can help in building performance decisions such as the selection of suitable materials, the analysis of thermal comfort, and the insulation.
The stated aspects all are determined using data-driven decisions by either using machine learning, optimization, or analysis software. Thus, having an integrated 3D heat transfer tool incorporated from the preliminary design phase to the schematic phase is essential to ultimately reduce the building's operational and embodied carbon emissions. 

Future work in this field includes in this field involves packaging the \gls{GH} automation workflow into a user-friendly plug-in. This plug-in would simplify the process of simulating 3D heat transfer, allowing users not only to build the case and create the mesh, but also to run any \gls{OF} simulations without the need to manually write text files. By offering this plug-in, architects, and engineers will be able to easily leverage the potential of \gls{OF} for their projects, facilitating broader adoption and exploration of advanced simulation techniques. Another potential is to expand the current workflow into a comprehensive 3D heat transfer toolkit. This toolkit would allow easier simulations to view heat transfer throughout the building, providing a more holistic approach to building thermal analysis. Moreover, \gls{OF} is capable of simulating to predict mold growth or condensation by changing the solver but using this same workflow. Finally, users can modify the solver inputs specifically the system file to include radiation in the simulation to have precise results in the cases where short- and long-wave radiations are critical. 








\section*{Research Outcomes}
First, this research was supported by a Microgrant from the Kendeda Building Advisory Board  \cite{kendeda} where funds were used to purchase the U-value measurement kit.\footnote{\url{https://web.archive.org/web/20240423144814/https://research.gatech.edu/micro-research-grants-awarded}}
The outcomes of this simulation experiment were presented at the Kendeda Microgrant Symposium in Spring 2024. 
A condensed version of this thesis has been accepted for submission in the 2024 International Building Physics conference proceedings in Toronto, Canada \cite{ibpc}. The full workflow implemented to simulate for 3D heat transfer can be found on GitHub\footnote{https://github.com/kastnerp/MT-3D-Heat-Transfer-Analysis-in-Architectural-Modeling}.
\todo{make this bullet point list and add in front under summary}




%\vspace{0.15in}
%\small
%\nomenclature{\(CHT\)}{Conjugate Heat Transfer}
%\nomenclature{\(PCM\)}{Phase Changing Materials}
%\nomenclature{\(CAD\)}{Computer-aided design}
%\nomenclature{\(ASHRAE\)}{American Society of Heating, Refrigerating and Air-Conditioning Engineers}
%\nomenclature{\(C_p\)}{Specific heat capacity, \si{\joule\per\kilogram\per\kelvin}}
%\nomenclature{\(d\)}{Length, \si{\metre}}
%\nomenclature{\(h\)}{Heat transfer coefficient, \si{\watt\per\metre\squared\per\kelvin}}
%\nomenclature{\(k\)}{Thermal conductivity, \si{\watt\per\metre\per\kelvin}}
%\nomenclature{\(\Psi\)}{Linear thermal transm. coeff., \si{\watt\per\metre\per\kelvin}}
%\nomenclature{\(q\)}{Heat flux density, \si{\watt\per\metre\squared}}
%\nomenclature{\(U\)}{Thermal transmittance, \si{\watt\per\metre\squared\per\kelvin}}
%\nomenclature{\(T\)}{Temperature, \si{\celsius\kelvin}}
%\nomenclature{\(x,\ y,\ z\)}{Cartesian coordinates, \si{\metre}}
