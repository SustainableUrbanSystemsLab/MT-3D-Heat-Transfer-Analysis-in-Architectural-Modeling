%\clearpage
\chapter{Conclusion}
This thesis presents a noticeable advancement in the field of sustainable architecture and building physics by introducing an automated integration between GH, Rhino, and OF to present thermal performance analysis in the architectural design process. \todo{this sounds like ChatGPT, just say what you have done. You built a workflow to simulate 3D heat transfer problems for architectural applications.}
By utilizing OF, this research fills a critical gap in current design practices and is capable of improving energy-efficient building design and data-driven material selection. 
Future work in this field includes packaging the GH automation to provide a user-friendly GH plug-in. Leveraging advanced computational tools like OpenFOAM and integrating them seamlessly into existing design software offers a practical solution to a challenge and provides architects with the means to make informed decisions easier and faster. \todo{please end with future work. Did you look at my research seminar material?}
Decisions include, suitable material selection, thermal comfort analysis, WWR, orientation, insulation, and many more. \todo{list all important ones, don't say many more.}
The stated aspects all are determined using data-driven decisions by either using machine learning, optimization, or analysis software. Thus, having an integrated 3d heat transfer tool incorporated from the preliminary design phase to the schematic phase is essential to ultimately reduce the building's operational and embodied carbon emissions. 

\section{Research Outcomes}
First, this research won the Kendeda Microgrant featured in \cite{kendeda} where the funds were used to purchase the U-value measurement kit. \todo{add URL to microgrant website}
The outcomes of this simulation experiment were presented at the XXX in Spring 2024. 
A condensed version of this thesis was has been accepted for submission in the 2024 International Building Physics conference proceedings in Toronto, Canada \cite{ibpc}. 
\todo{add the micro grant support also for the 2D chapter.}

%\vspace{0.15in}
%\small
%\nomenclature{\(CHT\)}{Conjugate Heat Transfer}
%\nomenclature{\(PCM\)}{Phase Changing Materials}
%\nomenclature{\(CAD\)}{Computer-aided design}
%\nomenclature{\(ASHRAE\)}{American Society of Heating, Refrigerating and Air-Conditioning Engineers}
%\nomenclature{\(C_p\)}{Specific heat capacity, \si{\joule\per\kilogram\per\kelvin}}
%\nomenclature{\(d\)}{Length, \si{\metre}}
%\nomenclature{\(h\)}{Heat transfer coefficient, \si{\watt\per\metre\squared\per\kelvin}}
%\nomenclature{\(k\)}{Thermal conductivity, \si{\watt\per\metre\per\kelvin}}
%\nomenclature{\(\Psi\)}{Linear thermal transm. coeff., \si{\watt\per\metre\per\kelvin}}
%\nomenclature{\(q\)}{Heat flux density, \si{\watt\per\metre\squared}}
%\nomenclature{\(U\)}{Thermal transmittance, \si{\watt\per\metre\squared\per\kelvin}}
%\nomenclature{\(T\)}{Temperature, \si{\celsius\kelvin}}
%\nomenclature{\(x,\ y,\ z\)}{Cartesian coordinates, \si{\metre}}
