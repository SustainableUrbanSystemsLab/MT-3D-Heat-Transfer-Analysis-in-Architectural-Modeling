%!TEX root = ../thesis.tex
\chapter{Introduction and Background}


Climate change is one of the most significant challenges of our time and has severe consequences for the environment, society, and economy. The alarming increase in global land temperature, which has increased by approximately 40\% \cite{glb} recently, highlights the importance of addressing this issue. At the same time, buildings are major contributors to carbon emissions, emitting approximately 40 gigatons of $CO_2$ from operational carbon annually. To maintain the limit of global warming of 1.5 °C, it is necessary to reduce emissions by 10 gigatons of $CO_2$ per year, which is not feasible in the current state \cite{ipcc}.



Focusing on the architectural and construction fields, architects and engineers refer to \gls{ASHRAE} standards to comply with the minimum required insulation according to the location of a project. 
In adhering to these standards, there is a potential to optimize the selection of materials toward a lower energy demand and to increase thermal comfort. Despite these recommendations, the process of modeling 3D heat transfer in buildings is a complex and long process due to the lack of free software available. 

This thesis aims to bridge the gap between providing architects with an easy-to-use 3D heat transfer workflow that is integrated into architect design software, such as Rhino\, \&\, \gls{GH}. Here, we used \gls{OF} / \gls{GH} to construct an envelope segment from \textit{ISO 10211:2007}
\cite{ISO}, then calculate heat transfer, and assess whether the validated case complies with our results. 



\section{Problem Statement}
Simulating 3D heat transfer is challenging due to the complexity of differential equations as a result of heat flow and transfer in 3D spaces. 
This complexity creates obstacles in achieving accurate simulations, which are critical in several applications. In the architectural field, the importance of 3D heat transfer simulations is significant in analyzing thermal comfort and thermal behavior of materials. The 3D simulation also has a direct impact on building design, affecting everything from the building envelope to material choices, and even the comfort and safety of occupants. The simulation also offers a detailed description of the modeled state, where thermal bridges and complex geometries are represented precisely.
However, existing software and simulation tools in architecture are generally limited to 2D heat transfer simulations, which cannot fully show the complexities represented in 3D thermal dynamics. Tools such as HEAT2, HTFlux, and THERM, widely used in the industry, provide only a partial view of the thermal environment, leading to potential inaccuracies in architectural planning or retrofitting.

Due to these limitations, the research question is the following: What approaches can be used to design and validate a workflow that simulates 3D heat transfer in architectural contexts? How can this workflow be seamlessly integrated into existing architectural modeling software? These questions aim to bridge the gap in current simulation technology by investigating the creation of a more advanced and reliable 3D simulation solution for architectural projects.

\section{Research Hypothesis}
This research is centered on the following hypothesis.
The development and validation of an integrated 3D heat transfer workflow to reduce errors in architectural thermal analysis, offering an easier approach to architectural decision-making.

\subsection{Relevance}
%\todo{applied relevance might not be a word}
\begin{itemize}
    \item The capacity to simulate both steady-state and transient heat transfer in \gls{3D} geometries using \gls{OF} is crucial for accurate modeling in architectural contexts. The simulations are then validated with the \gls{ISO} and experimental design data.
   
    \item To simplify the OF simulations, it'll be integrated with \gls{Rhino} and \gls{GH} to facilitate the \gls{CFD} inclusion into architectural modeling.
\end{itemize}


\subsection{Functional Benefits}
\begin{itemize}
   \item Integrating 3D heat transfer simulation workflows directly into architectural modeling software will enhance the efficiency of architectural designs by reducing costs and simplifying the process.
 
   \item Automating the generation of OpenFOAM text files and developing an open-source workflow will reduce the time and effort required for simulations. Further testing is necessary to fully assess its impact on the speed of design iterations.

\end{itemize}

\subsection{Practical Applications}
\begin{itemize}
\item This project presents a seamless 3D heat transfer workflow that could be included in the design optimization process to reduce carbon emissions.
%\todo{You are not doing that}
\item  Utilizing a single integrated workflow for heat transfer simulations in architectural design eliminates software costs. This will reduce the need for multiple tools.

%\todo{please write as actual hypothesis}
\item The open-source workflow is accessible and motivates and encourages any designer to reduce operational and embodied carbon. This will be further explored in future work, as the current phase of research does not verify these impacts and requires additional investigation.
%\todo{You didn't test that, so you need to hint at that in the future work. Need to revisit and say you cannot verify at this stage}



\end{itemize}
\section{Research Aim \& Objectives}
\subsection{Research Aim}
The main objective of the research is to develop and validate a comprehensive 3D heat transfer simulation workflow integrated into architectural modeling processes. To achieve this research goal, the following specific objectives have been established.

\subsection{Research Objectives}
\begin{itemize}
    \item Select a suitable case study to validate the simulation results and develop a connection between Rhino, \gls{GH}, and \gls{OF} to streamline the architectural modeling workflow. 
    \item  Create a 3D heat transfer simulation workflow using OpenFOAM and ensure the accuracy of the 3D simulation results through comprehensive validation.
    \item Test the developed workflow by incorporating real-time heat flux sensor readings.
 
\end{itemize}







\section{Thesis Outline}
This thesis is structured as follows:
\begin{itemize}
    \item \textit{Chapter 2} showcases the history of heat transfer in buildings, literature review, heat transfer processes, and computational fluid dynamics.
    \item \textit{Chapter 3} illustrates three approaches to finding the heat flux of a brick wall; manual calculation, sensor reading, and 2D heat transfer simulation. These approaches are presented to highlight the shortcomings of current resources. 
    \item \textit{Chapter 4} focuses on the methodology of the 3D heat transfer workflow, identifies an appropriate validation case study, and discusses the solver and process in detail.
    \item \textit{Chapter 5} ensures the compliance of the objectives and provides an outlook with suggestions for an expansion of this project.
\end{itemize}