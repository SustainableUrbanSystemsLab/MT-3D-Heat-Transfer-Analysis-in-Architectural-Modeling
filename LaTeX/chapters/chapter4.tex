%\clearpage
\chapter{Conclusion}
This thesis presents an automated integration between  \gls{GH}, Rhino, and \gls{OF} workflow to simulate 3D heat transfer in architectural applications. 
Using OF, this research fills a critical gap in current design practices and is capable of improving energy-efficient building design and data-driven material selection. 

The 3D heat transfer workflow can help improve decision-making in building performance by contributing to the selection of appropriate materials, analyzing thermal comfort, and optimizing insulation.
Thus, having an integrated 3D heat transfer workflow incorporated from the preliminary design phase to the schematic phase is essential to ultimately reduce the building's operational and embodied carbon emissions. 

Future work in this field involves packaging the \gls{GH} automation workflow into a user-friendly plug-in. 
A plug-in would simplify the process of simulating 3D heat transfer for other modelers, allowing them not only to build the case and create the mesh, but also to run any \gls{OF} simulations without the need to manually write text files. 
By offering this plug-in, modelers will be able to easily leverage the potential of \gls{OF} for their projects, facilitating the wider adoption and exploration of advanced simulation techniques. 
Another potential is to expand the current workflow into a comprehensive 3D heat transfer plug-in. 
This plug-in would allow for easier simulations to view heat transfer throughout the building, providing a more holistic approach to building thermal analysis. 

 Also, the presented workflow can be adapted by changing the solver to predict condensation through hygrothermal simulations to assess potential mold growth risks.

Finally, future work also includes modifying the solver inputs, specifically the system file, to include radiation in the simulation to have precise results in cases where short- and long-wave radiations are critical. 







%\vspace{0.15in}
%\small
%\nomenclature{\(CHT\)}{Conjugate Heat Transfer}
%\nomenclature{\(PCM\)}{Phase Changing Materials}
%\nomenclature{\(CAD\)}{Computer-aided design}
%\nomenclature{\(ASHRAE\)}{American Society of Heating, Refrigerating and Air-Conditioning Engineers}
%\nomenclature{\(C_p\)}{Specific heat capacity, \si{\joule\per\kilogram\per\kelvin}}
%\nomenclature{\(d\)}{Length, \si{\metre}}
%\nomenclature{\(h\)}{Heat transfer coefficient, \si{\watt\per\metre\squared\per\kelvin}}
%\nomenclature{\(k\)}{Thermal conductivity, \si{\watt\per\metre\per\kelvin}}
%\nomenclature{\(\Psi\)}{Linear thermal transm. coeff., \si{\watt\per\metre\per\kelvin}}
%\nomenclature{\(q\)}{Heat flux density, \si{\watt\per\metre\squared}}
%\nomenclature{\(U\)}{Thermal transmittance, \si{\watt\per\metre\squared\per\kelvin}}
%\nomenclature{\(T\)}{Temperature, \si{\celsius\kelvin}}
%\nomenclature{\(x,\ y,\ z\)}{Cartesian coordinates, \si{\metre}}
