\thispagestyle{empty}
\begin{summary}
\enlargethispage{2em}

As the global focus on sustainable building practices intensifies, architects face the challenge of designing structures that meet certain aesthetic and functional criteria while minimizing energy consumption. 
One critical aspect of achieving energy-efficient buildings is the selection of appropriate building materials with optimal thermal properties. 



The tools and software to simulate 2D heat transfer are available, but often limited in their set of features and cost-prohibitive. The limitations of 2D heat transfer are the inability to simulate and explore complex geometry, corners, and full building envelope analysis.
The integration of 3D thermal performance analysis into the architectural design process is an even more complex and underdeveloped area. 


This thesis aims to address this gap by exploring the use of OpenFOAM to develop a user-friendly tool to simulate building-related heat transfer problems.
The outcomes of this thesis aim to empower architects to make informed decisions about material selection, and their impact on energy efficiency, by seamlessly embedding it into the Rhino \& Grasshopper CAD environment. 

\vspace{-0.5em}

\subsection*{Thesis outcomes}


\begin{itemize}[topsep=0pt,itemsep=0pt,partopsep=0pt, parsep=0pt]
    %\setlength\itemsep{-1em}

    \item  The full workflow implemented here can be found on GitHub\footnote{\url{https://bit.ly/3UzNFoL}}.

    \item This research was supported by a Microgrant   \cite{kendeda} from the Kendeda Building Advisory Board, and funds were used to purchase the U-value measurement kit\footnote{\url{https://web.archive.org/web/20240423144814/https://research.gatech.edu/micro-research-grants-awarded}}. The outcomes of this simulation experiment were presented at the Kendeda Microgrant Symposium in Spring 2024.
    
    \item A condensed version of this thesis has been accepted for submission in the proceedings of the 2024 International Building Physics Conference in Toronto, Canada \cite{MPIbpc}.
    
    

\end{itemize}

\end{summary} 


